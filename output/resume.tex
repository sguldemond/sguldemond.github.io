% Copyright 2013 Christophe-Marie Duquesne <chmd@chmd.fr>
% Copyright 2014 Mark Szepieniec <http://github.com/mszep>
% 
% ConText style for making a resume with pandoc. Inspired by moderncv.
% 
% This CSS document is delivered to you under the CC BY-SA 3.0 License.
% https://creativecommons.org/licenses/by-sa/3.0/deed.en_US

\startmode[*mkii]
  \enableregime[utf-8]  
  \setupcolors[state=start]
\stopmode

\setupcolor[hex]
\definecolor[titlegrey][h=000000]
\definecolor[sectioncolor][h=00008B]
\definecolor[rulecolor][h=9cb770]

% Enable hyperlinks
\setupinteraction[state=start, color=sectioncolor]

\setuppapersize [A4][A4]
\setuplayout    [width=middle, height=middle,
                 backspace=20mm, cutspace=0mm,
                 topspace=10mm, bottomspace=20mm,
                 header=0mm, footer=0mm]

%\setuppagenumbering[location={footer,center}]

\setupbodyfont[11pt, helvetica]

\setupwhitespace[medium]

\setupblackrules[width=0mm, color=rulecolor]

\setuphead[chapter]      [style=\tfd]
\setuphead[section]      [style=\tfd\bf, color=titlegrey, align=middle]
\setuphead[subsection]   [style=\tfb\bf, color=sectioncolor, align=right,
                          before={\leavevmode\blackrule\hspace}]
\setuphead[subsubsection][style=\bf]

\setuphead[chapter, section, subsection, subsubsection][number=no]

%\setupdescriptions[width=10mm]

\definedescription
  [description]
  [headstyle=bold, style=normal,
   location=hanging, width=18mm, distance=14mm, margin=0cm]

\setupitemize[autointro, packed]    % prevent orphan list intro
\setupitemize[indentnext=no]

\setupfloat[figure][default={here,nonumber}]
\setupfloat[table][default={here,nonumber}]

\setuptables[textwidth=max, HL=none]
\setupxtable[frame=off,option={stretch,width}]

\setupthinrules[width=15em] % width of horizontal rules

\setupdelimitedtext
  [blockquote]
  [before={\setupalign[middle]},
   indentnext=no,
  ]


\starttext

\section[title={Stan Guldemond},reference={stan-guldemond}]

\subsection[title={Full stack developer with a passion for creating easy
to use, scalable, readable, problem solving
software},reference={full-stack-developer-with-a-passion-for-creating-easy-to-use-scalable-readable-problem-solving-software}]

{\externalfigure[../assets/stan-amsterdam-2019_300x300.jpg]}

\startitemize[packed]
\item
  \useURL[url1][mailto:stanguldemond@gmail.com][][stanguldemond@gmail.com]\from[url1]
\item
  +31625405664
\item
  Amsterdam, The Netherlands
\item
  \useURL[url2][https://www.linkedin.com/in/stan-guldemond-56291b120/][][LinkedIn]\from[url2],
  \useURL[url3][https://github.com/sguldemond/][][GitHub]\from[url3],
  \useURL[url4][https://gitlab.com/users/stanguldemond/contributed][][GitLab]\from[url4],
  \useURL[url5][https://stackshare.io/sguldemond][][Stackshare]\from[url5]
\stopitemize

\thinrule

\subsection[title={Stack},reference={stack}]

\startitemize[packed]
\item
  Python 3,
  \useURL[url6][https://flask.palletsprojects.com/][][Flask]\from[url6],
  \useURL[url7][fastapi.tiangolo.com/][][FastAPI]\from[url7],
  \useURL[url8][https://www.sqlalchemy.org/][][SQLAlchemy]\from[url8],
  \useURL[url9][https://pypi.org/project/aio-pika/][][aio-pika]\from[url9],
  \useURL[url10][https://docs.python.org/3/library/typing.html][][typing]\from[url10],
  \useURL[url11][https://docs.pytest.org][][pytest]\from[url11], PEP 8,
  virtual environments
\item
  PostgreSQL,
  \useURL[url12][https://postgis.net/][][PostGIS]\from[url12],
  (materialized) views, indexes,
  \useURL[url13][https://www.pgadmin.org/][][pgAdmin]\from[url13]
\item
  Docker, docker-compose, Docker networks
\item
  Linux, Ubuntu Desktop & Server, Git, bash, zsh, shell script, crontab
\item
  GitLab CI/CD pipelines
\item
  RabbitMQ, Nginx, Celery, VueJS, Node, VSCode, PyCharm, Git Flow
\stopitemize

\thinrule

\subsection[title={Experience},reference={experience}]

2018-now: {\bf Full Stack Developer} // CTO Innovatieteam, City of
Amsterdam (Amsterdam, The Netherlands)

\starttyping
Currently developing human centric tools, with a data driven approach, supporting the maintenance of the public space in Amsterdam. Working in a self managed multi disciplinary team with a focus to make things work, see 'Projects'.

Previous activities include:
- Designing + developing an Attribute Based Digital Identity prototype with a focus on data ownership and zero knowledge proof as part of the European Commission funded Decode program.
- Maintaining & expending our infrastructure capabilities using Gutter (see Projects).
\stoptyping

2017: {\bf Software Engineer} // Ceron IT Solutions (Vught, The
Netherlands)

\starttyping
- Developing and maintaining web application with Java EE backend
- Implementing PrimeFaces framework frontend 
\stoptyping

2016: {\bf Student Assistent} // Fontys University of Applied Sciences
(Eindhoven, The Netherlands)

\starttyping
Assisting part-time students, between the ages of 25 and 50, twice a week in the subject of object oriented programming focussing on the basics of programming in C# and software design. 
\stoptyping

2016: {\bf Teacher} // SintLucas (Eindhoven, The Netherlands)

\starttyping
Part of my bachelor's degree in software engineering. I got to teach young adults the basics of object oriented programming in C# and assist them in various other skills. 
\stoptyping

\thinrule

\subsection[title={Education},reference={education}]

2015-2019: {\bf Bachelor's Degree, Software Engineering} // Fontys
University of Applied Sciences (Eindhoven, The Netherlands)

\starttyping
With software engineering I've learned to design software in an efficient and maintainable way. Next to this main focus I've had a introduction into teaching young adults and a introduction into designing software for mobile devices.
\stoptyping

2011-2013: {\bf Applied Psychology} // Fontys University of Applied
Sciences (Eindhoven, The Netherlands)

\starttyping
Learning about the working of the human mind, conversation techniques and analysing social issues. Initiated a study group with classmates.
\stoptyping

2007-2011: {\bf MBO BOL-4, Graphic Design} // Grafisch Lyceum Rotterdam
(Rotterdam, The Netherlands)

\starttyping
Graphic Design, Adobe Creative Suite, Desktop Publishing
\stoptyping

\thinrule

\subsection[title={Projects},reference={projects}]

{\bf Object Detection Kit}: Open source object recognition tool for
municipalities to spot misplaced garbage in the streets and act on it
quickly, even before citizens notice. My role and responsibilities:

\starttyping
- Lead developer: Code reviews, architecture design, communication with stackholders, user tests, hiring new developers, documentation
- Development: Python, VueJS, PostgreSQL
- DevOps: GitLab CI, docker-compose, Ubuntu Server
- Hardware: Maintanence GPU servers for image recognition
\stoptyping

\startitemize[packed]
\item
  \useURL[url14][https://www.odk.ai][][Project website]\from[url14]
\item
  \useURL[url15][https://www.gitlab.com/odk/odk-stack][][Open source
  repository]\from[url15]
\item
  \useURL[url16][https://www.gitlab.com/odk][][All ODK
  projects]\from[url16]
\stopitemize

{\bf Decode Digital Identity}: Prototype for attribute based digital
identity utilizing the data on a NFC readable passport. The different
elements of this project I created:

\starttyping
- Passport scanner: Controlling NFC reader, camera and display using Python
- Mobile app: VueJS PWA
- Backend: Python Flask, Zenroom
\stoptyping

This project made money avaliable for continues research and development
of the use of digital identity in the City of Amsterdam.

\startitemize[packed]
\item
  \useURL[url17][https://www.vimeo.com/384562767][][Me presenting the
  project in Turin, Italy]\from[url17]
\item
  \useURL[url18][https://www.github.com/Amsterdam/decode_passport_scanner][][GitHub
  repository]\from[url18]
\item
  \useURL[url19][https://www.youtube.com/watch?v=p1KLvwV7oIM][][Quick
  demo]\from[url19]
\item
  \useURL[url20][https://www.decodeproject.eu][][Decode
  Project]\from[url20]
\stopitemize

{\bf Gutter}: Python based API generator, with build in ETL tooling,
real time analytics, message queuing and more utilizing the power of
PostgreSQL.

\starttyping
Build an analytics engine using Celery workers. It scheduled analytics jobs (Python pandas) and persisted the output data in a database (PostgreSQL) then automatically generated and deployed REST APIs (Flask). think Apache Airflow x Google Firebase.
\stoptyping

\startitemize[packed]
\item
  \useURL[url21][https://www.github.com/amsterdam/gutter][][Outdated
  repository]\from[url21]
\stopitemize

\thinrule

\subsection[title={Miscellaneous},reference={miscellaneous}]

{\bf Languages}:

\startitemize[packed]
\item
  Dutch (mother tongue)
\item
  English (professional, daily speaker)
\item
  Polish (beginner)
\stopitemize

{\bf Hobbies}

\startitemize[packed]
\item
  Linux, coding useful tools for myself
\item
  Climbing, bouldering, long distance cycling, yoga, camping
\item
  Spirituality, mental health, films, music, reading, writing, drawing
\stopitemize

\thinrule

\stoptext
